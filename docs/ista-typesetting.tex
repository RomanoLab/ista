% ista-options.tex  (plain TeX, not LaTeX)
% Compile with: pdftex ista-options.tex   (or: tex ista-options.tex -> dvi)

\magnification=\magstep1
\parindent=0pt
\parskip=6pt
\headline={}
\footline={\hfil\tenrm ista typesetting options\hfil\folio}

% --- Fonts (Computer Modern family) ---
\font\tenrm=cmr10
\font\tenbf=cmbx10
\font\tentt=cmtt10
\font\tensf=cmss10
\font\tensc=cmcsc10
\font\bigrm=cmr10 scaled 2000
\font\bigbf=cmbx10 scaled 2000
\font\bigtt=cmtt10 scaled 2000
\font\bigsf=cmss10 scaled 2000
\font\bigsc=cmcsc10 scaled 2000

\tenrm

% --- Minimal "verbatim" for code display ---
\def\begincode{%
  \par\smallskip
  \hrule\smallskip
  \begingroup
    \parindent=0pt
    \tentt
    \catcode`\\=12 \catcode`\{=12 \catcode`\}=12
    \catcode`\$=12 \catcode`\&=12 \catcode`\#=12
    \catcode`\%=12 \catcode`\~=12 \catcode`\_=12
    \obeylines \obeyspaces
}
\def\endcode{%
  \endgroup
  \smallskip\hrule
  \par\bigskip
}

% --- A little section helper ---
\def\Option#1#2#3{%
  {\tenbf #1}\par
  Rendered (normal size): #2\par
  Rendered (large): {\bigrm #2}\par
  Code:\par
  \begincode
#3
  \endcode
}

% --- Definitions of the options ---
\def\istaPlain{{\tenrm ista}}

\def\istaSC{{\tensc ista}}
\def\istaSCcaps{{\tensc ISTA}}

\def\istaTT{{\tentt ista}}
\def\istaTTtracked{{\tentt i\kern0.05em s\kern0.05em t\kern0.05em a}}

\def\istaSF{{\tensf ista}}

\def\istaMathRM{$\rm ista$}

% TeX-logo-style kerning + lowering (two variants)
\def\istaKernLowerS{{\tenrm i\kern-0.08em\lower0.20ex\hbox{s}\kern-0.06em t\kern-0.02em a}}
\def\istaKernLowerT{{\tenrm i\kern-0.06em s\kern-0.04em\lower0.18ex\hbox{t}\kern-0.02em a}}

% "Graph node" dot on i (dotless i + raised bullet)
\def\istaNodeDot{{\tenrm \i\kern-0.02em\raise0.55ex\hbox{$\scriptstyle\bullet$}\kern0.06em sta}}

% A lightweight "conversion" mark: a ↔ above the word (plain-TeX-friendly)
\def\istaConvert{%
  {\setbox0=\hbox{\tenrm ista}%
   \hbox{\vbox{\offinterlineskip
     \halign{\hfil##\hfil\cr
       \hbox{$\leftrightarrow$}\cr
       \noalign{\kern1pt}
       \box0\cr}}}}%
}

% Tighten the s-t gap (CM Roman doesn't have an st ligature, but this reads "designed")
\def\istaTightST{{\tenrm i s\kern-0.10em t a}}

% --- Title ---
{\bigbf ista} {\tenrm typesetting comparison (plain TeX)}\par
\smallskip
This file shows several “Knuth-adjacent” ways to typeset the name.\par
\bigskip
\hrule
\bigskip

% --- Render each option + show the code used ---
\Option
  {1) Plain roman (baseline)}
  {\istaPlain}
  {\def\istaPlain{{\tenrm ista}}}

\Option
  {2) Small caps}
  {\istaSC}
  {\def\istaSC{{\tensc ista}}}

\Option
  {3) Small caps (all-caps shape, still small-caps glyphs)}
  {\istaSCcaps}
  {\def\istaSCcaps{{\tensc ISTA}}}

\Option
  {4) Monospace / typewriter}
  {\istaTT}
  {\def\istaTT{{\tentt ista}}}

\Option
  {5) Monospace with subtle tracking}
  {\istaTTtracked}
  {\def\istaTTtracked{{\tentt i\kern0.05em s\kern0.05em t\kern0.05em a}}}

\Option
  {6) Sans serif}
  {\istaSF}
  {\def\istaSF{{\tensf ista}}}

\Option
  {7) Math roman token (operator-ish feel)}
  {\istaMathRM}
  {\def\istaMathRM{$\rm ista$}}

\Option
  {8) TeX-logo trick variant A (lower the ‘s’)}
  {\istaKernLowerS}
  {\def\istaKernLowerS{{\tenrm i\kern-0.08em\lower0.20ex\hbox{s}\kern-0.06em t\kern-0.02em a}}}

\Option
  {9) TeX-logo trick variant B (lower the ‘t’)}
  {\istaKernLowerT}
  {\def\istaKernLowerT{{\tenrm i\kern-0.06em s\kern-0.04em\lower0.18ex\hbox{t}\kern-0.02em a}}}

\Option
  {10) “Knowledge graph” dot on the i (node dot)}
  {\istaNodeDot}
  {\def\istaNodeDot{{\tenrm \i\kern-0.02em\raise0.55ex\hbox{$\scriptstyle\bullet$}\kern0.06em sta}}}

\Option
  {11) Conversion mark (↔ above the word)}
  {\istaConvert}
  {\def\istaConvert{{\setbox0=\hbox{\tenrm ista}\hbox{\vbox{\offinterlineskip
    \halign{\hfil##\hfil\cr
      \hbox{$\leftrightarrow$}\cr
      \noalign{\kern1pt}
      \box0\cr}}}}}}

\Option
  {12) Tightened “s-t” spacing}
  {\istaTightST}
  {\def\istaTightST{{\tenrm i s\kern-0.10em t a}}}

\bye
